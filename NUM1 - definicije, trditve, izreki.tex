\documentclass[11pt]{article}
\usepackage[utf8]{inputenc}
\usepackage[slovene]{babel}

\usepackage{amsthm}
\usepackage{amsmath, amssymb, amsfonts}
\usepackage{relsize}
\usepackage{listings}
\lstset{
	basicstyle=\ttfamily,
	mathescape
}

\theoremstyle{definition}
\newtheorem{definicija}{Definicija}[section]

\newtheorem{lema}{Lema}
\newtheorem{trditev}{Trditev}
\newtheorem{izrek}{Izrek}
\newtheorem*{dokaz}{Dokaz}
\newtheorem*{algoritem}{Algoritem}
\newtheorem*{posledica}{Posledica}
\newtheorem*{metoda}{Metoda}

\title{Numerične metode 1 - definicije, trditve in izreki}
\author{Oskar Vavtar}
\date{2020/21}

\begin{document}
\maketitle
\pagebreak
\tableofcontents
\pagebreak

% #################################################################################################

\section{NUMERIČNO RAČUNANJE}
\vspace{0.5cm}

% *************************************************************************************************

\subsection{Uvod}
\vspace{0.5cm}

\begin{definicija}[Napaka]

Pri numeričnem računanju izračunamo numerični približek za točno rešitev. Razlika med približkom in točno vrednostjo je \textit{napaka} približka. Ločimo \textit{absolutno} in \textit{relativno} napako.

\begin{itemize}
	\item absolutna napaka $=$ približek $-$ točna vrednost
	\item relativna napaka $=$ $\frac{\text{absolutna napaka}}{\text{točna vrednost}}$
\end{itemize}

Naj bo $x$ točna vrednost, $\hat{x}$ pa približek za $x$.

\begin{itemize}
	\item Če je $\hat{x} = x + d_a$, potem je $d_a = \hat{x} - x$ \textit{absolutna napaka}.
	\item Če je $\hat{x} = x(1 + d_r)$, potem je $d_r = \frac{\hat{x} - x}{x}$ \textit{relativna napaka}.
\end{itemize}

\end{definicija}
\vspace{0.5cm}

% *************************************************************************************************

\subsection{Premična pika}
\vspace{0.5cm}

\begin{definicija}

Velja fl$(x) = x(1 + \delta)$ za $|\delta| \leq u$, kjer je
$$u = \frac{1}{2}b^{1-t}$$
\textit{osnovna zaokrožitvena napaka}:

\begin{itemize}
	\item single: $u = 2^{-24} = 6 \times 10^{-8}$
	\item double: $u = 2^{-53} = 1 \times 10^{-16}$
\end{itemize}

\end{definicija}
\vspace{0.5cm}

\begin{izrek}

Če za število $x$ velja, da $|x|$ leži na intervalu med najmanjšim in največjim \textit{pozitivnim predstavljivim normaliziranim} številom, potem velja
$$\frac{|\text{fl}(x) - x|}{|x|} \leq u.$$

\end{izrek}
\vspace{0.5cm}

% *************************************************************************************************

\subsection{Občutljivost problema}
\vspace{0.5cm}

\begin{definicija}[Občutljivost]

Če se rezultat pri majhni spremembi argumentov (\textit{motnji} oz. \textit{perturbaciji}) ne spremeni veliko, je problem \textit{neobčutljiv}, sicer pa je \textit{občutljiv}.

\end{definicija}
\vspace{0.5cm}

% *************************************************************************************************

\subsection{Vrste napak pri numeričnem računanju}
\vspace{0.5cm}

\begin{definicija}

Pri numeričnem računanje se pojavijo 3 vrste napak:
\begin{enumerate}
	\item \textsc{Neodstranljive napake}: 
	Npr. ko podatek ni predstavljivo število. Namesto $y = f(x)$ lahko v najboljšem primeru izračunamo $\overline{y} = f(\overline{x})$, kjer je $\overline{x}$ najbližje predstavljivo število. 
	$$D_n = y - \overline{y} = f(x) - f(\overline{x})$$
	\item \textsc{Napaka metode}:
	Npr. ko na voljo nimamo željene operacije. Namesto $f(\overline{x})$ potem izračunamo $\widetilde{y}=g(\overline{x})$, kjer je $g(x)$ približek za $f(x)$, kjer znamo vrednost $g$ izračunati s končnim številom operacij. 
	$$D_m = \overline{y} - \widetilde{y} = f(\overline{x}) - g(\overline{x})$$
	\item \textsc{Zaokrožitvene napake}:
	Pri izračunu $\widetilde{y}=g(\overline{x})$ lahko pri vsaki osnovni operaciji pride do zaokrožitvene napake, zato na koncu kot numeričen rezultat dobimo $\widehat{y}$.
	$$D_z = \widetilde{y} - \widehat{y}$$
\end{enumerate}

Skupna napaka:
$$D = D_n + D_m + D_z$$

V splošnem lahko ocenimo:
$$|D| \leq |D_n| + |D_m| + |D_z|$$
\end{definicija}
\vspace{0.5cm}

% *************************************************************************************************

\subsection{Stabilnost metode}
\vspace{0.5cm}

\begin{definicija}

Če metoda za $\forall x$ vrne $\widehat{y}$, ki je \textit{absolutno (relativno)} blizu točnemu $y$, je metoda \textit{direktno stabilna}.

Če metoda za $\forall x$ vrne tak $\widehat{y}$, da $\exists \widehat{x}$ absolutno (relativno) blizu $x$, da je $\widehat{y}=f(\widehat{x})$ (točno), je metoda \textit{obratno stabilna}. \\

\noindent V splošnem:
$$|\text{direktna napaka}| \leq |\text{občutljivost}| \cdot |\text{obratna napaka}|$$

\end{definicija}
\vspace{0.5cm}

% *************************************************************************************************

\subsection{Analiza zaokrožitvenih napak}
\vspace{0.5cm}

\subsubsection{Produkt $n+1$ predstavljivih števil}

\begin{algoritem}

Dana so predstavljiva števila $x_0, x_1, \ldots, x_n$; \\ računamo $p = x_0 \cdot x_1 \cdot \ldots \cdot x_n$. \\

\noindent \textsc{Točno}:
\begin{lstlisting}
p$_\texttt{0}$ = x$_\texttt{0}$
i = 1,...,n
$~~~~$p$_\texttt{i}$ = p$_\texttt{i-1} \cdot$x$_\texttt{i}$
p = p$_\texttt{n}$
\end{lstlisting} 

\noindent \textsc{Numerično}:
\begin{lstlisting}
$\widehat{\texttt{p}}_\texttt{0}$ = x$_\texttt{0}$
i = 1,...,n
$~~~~$$\widehat{\texttt{p}}_\texttt{i}$ = $\widehat{\texttt{p}}_\texttt{i-1} \cdot$x$_\texttt{i}\cdot$(1 + $\delta_\texttt{i}$)$~~|\delta_\texttt{i}| \leq$ u
$\widehat{\texttt{p}}$ = $\widehat{\texttt{p}}_\texttt{n}$
\end{lstlisting}

\end{algoritem}
\vspace{0.5cm}
\pagebreak

\subsubsection{Skalarni produkt vektorjev dolžine $n$}

\begin{algoritem}

Imamo vektorje \textit{predstavljivih} števil $a = [a_1, \dots, a_n]^T$, $b = [b_1, \dots, b_n]^T$. Računamo $s = \langle b^T, a \rangle = \mathlarger{\sum_{i=1}^n a_i b_i}$.

\noindent \textsc{Točno}:
\begin{lstlisting}
s$_\texttt{0}$ = 0
i = 1,...,n
$~~~~$p$_\texttt{i}$ = a$_\texttt{i} \cdot$b$_\texttt{i}$
$~~~~$s$_\texttt{i}$ = s$_\texttt{i-1}$ + p$_\texttt{i}$
s = s$_\texttt{n}$
\end{lstlisting}

\noindent \textsc{Numerično}:
\begin{lstlisting}
$\widehat{\texttt{s}}_\texttt{0}$ = 0
i = 1,...,n
$~~~~\widehat{\texttt{p}}_\texttt{i}$ = a$_\texttt{i} \cdot$b$_\texttt{i} \cdot$(1 + $\alpha_i$)$~~~~~~~~|\alpha_i| \leq u$
$~~~~\widehat{\texttt{s}}_\texttt{i}$ = ($\widehat{\texttt{s}}_\texttt{i-1}$ + $\widehat{\texttt{p}}_\texttt{i}$)$\cdot$(1 + $\beta_i$)$~~|\beta_i| \leq u$
$\widehat{\texttt{s}}$ = $\widehat{\texttt{s}}_\texttt{n}$
\end{lstlisting}

\end{algoritem}
\vspace{0.5cm}

\begin{trditev}

Računanje skalarnega produkta je \textit{obratno} stabilno, ni pa \textit{direktno} stabilno.

\end{trditev}
\vspace{0.5cm}

% *************************************************************************************************

\pagebreak

% #################################################################################################

\section{NELINEARNE ENAČBE}
\vspace{0.5cm}

% *************************************************************************************************

\subsection{Uvod}
\vspace{0.5cm}

\begin{definicija}

Naj bo $\alpha$ ničla funkcije $f$, ki je \textit{zvezno odvedljiva} v okolici $\alpha$:
\begin{itemize}

	\item $f'(\alpha) \neq 0$: $\alpha$ je \textit{enostavna} ničla
	
	\item $f'(\alpha) = 0$: $\alpha$ je \textit{večkratna} ničla \\
	
	Če je $f$ $m$-krat zvezno odveljiva in
	$$f'(\alpha) = f''(\alpha) = \ldots = f^{(m-1)}(\alpha) = 0,~~f^{(m)} \neq 0,$$
	je $\alpha$ $m$-kratna ničla.

\end{itemize}

\end{definicija}
\vspace{0.5cm}

\begin{trditev}[Občutljivost ničel]

Naj bo $\alpha$ enostavna ničla. Če v okolici $x = \alpha$ obstaja inverzna funkcija $\alpha = f^{-1}(0)$ v bistvu "računamo" vrednost inverzne funkcije. Občutljivost je enaka absolutni vrednosti odvoda inverzna funkcije:
$$|(f^{-1})'(0)| = \frac{1}{|f'(f^{-1}(0))|} = \frac{1}{|f'(\alpha)|}.$$

Večkratno ničlo lahko izračunamo le z natančnostjo $u^{\frac{1}{m}}$, kjer je $m$ večkratnost ničle (za dvojno ničlo dobimo le polovico točnih decimalk, za trojno le tretjino...). 

\end{trditev}
\vspace{0.5cm}

% *************************************************************************************************

\subsection{Bisekcija}
\vspace{0.5cm}

\begin{izrek}

Če je $f$ realna zvezna funkcija na $[a, b]$ in je $f(a) \cdot f(b) < 0$, potem $\exists \xi \in (a, b)$, da je $f(\xi) = 0$.

\end{izrek}
\vspace{3cm}

\begin{algoritem}[Bisekcija]

Naj velja $f(a) \cdot f(b) < 0$ in $a < b$:
\begin{lstlisting}
e = b - a
while e > $\varepsilon$
$~~~~$e = e/2, c = a + e
$~~~~$if sign(f(c)) = sign(f(a))
$~~~~~~~~$a = c
$~~~~$else
$~~~~~~~~$b = c
\end{lstlisting}
\end{algoritem}
\vspace{0.5cm}

% *************************************************************************************************

\subsection{Navadna iteracija}
\vspace{0.5cm}

\begin{algoritem}[Navadna iteracija]
~\\
\begin{lstlisting}
izberi x$_\texttt{0}$
r = 0, 1, 2,...
$~~~~$x$_{\texttt{r+1}}$ = g(x$_{\texttt{r}}$)
\end{lstlisting}
~\\
Ustavitveni kriterij:
\begin{enumerate}
	\item[a)] $r > r_{max}$ (prekoračeno število korakov)
	\item[b)] $|r_{x+1} - r_x| < \varepsilon$
\end{enumerate}
\end{algoritem}
\vspace{0.5cm}

\begin{izrek}

Naj bo $\alpha = g(\alpha)$ in naj iteracijska  funkcija $g$ na intervalu $I = [\alpha - \delta, \alpha + \delta]$ za nek $\delta > 0$ zadošča Lipschitzovem pogoju
$$|g(x) - g(y)| \leq m|x - y| ~~\text{za}~~ x, y \in I, 0 \leq m < 1.$$
Potem za $\forall x_0 \in I$ zaporedje $x_{r+1} = g(x_r)$, $r = 0, 1, \ldots$ konvergira k $\alpha$ in velja
\begin{itemize}
	\item $|x_r - \alpha| ~\leq~ m^r \cdot |x_0 - \alpha|$
	\item $|x_{r+1} - \alpha| ~\leq~ \cfrac{m}{1-m} \cdot |x_{r+1} - x_r|$
\end{itemize}

\end{izrek}
\vspace{0.5cm}

\begin{posledica}

Naj bo $\alpha = g(\alpha)$, $g$ \textit{zvezno odvedljiva} in $|g'(\alpha)| < 1$. Potem $\exists \delta > 0$, da za $\forall x_0 \in [\alpha - \delta, \alpha + \delta]$ zaporedje $x_{r+1} = g(x_r)$ konvergira k $\alpha$.

\end{posledica}
\vspace{0.5cm}

\begin{definicija}

Naj zaporedje $\{ x_r \}$ konvergira proti $\alpha$ ($\lim_{r \rightarrow \infty} x_r = \alpha)$. Pravimo, da zaporedje konvergira z redom konvergence $p$, če obstaja limita
$$\lim_{r \rightarrow \infty} \frac{|x_{r+1} - \alpha|}{|x_r - \alpha|^p} = C > 0.$$

\end{definicija}
\vspace{0.5cm}

\begin{izrek}

Naj bo iteracijska funkcija $g$ $p$-krat \textit{zvezno odvedljiva} v okolici negibne točke $\alpha$. Če velja $g'(\alpha) = \ldots = g^{(p-1)}(\alpha) = 0$ in $g^{(p)}(\alpha) \neq 0$, potem zaporedje $x_{r+1} = g(x_r)$, $r = 0, 1, \ldots$, v okolici $\alpha$ konvergira z redom $p$. V primeru $p = 1$ mora za konvergenco veljati še $|g'(\alpha)| < 1$. 

\end{izrek}
\vspace{0.5cm}

% *************************************************************************************************

\subsection{Tangentna metoda}
\vspace{0.5cm}

\begin{metoda}

$$x_{r+1} = x_r - \frac{f(x_r)}{f'(x_r)}$$
Konvergenca:
$$|e_{r+1}| \approx C \cdot |e_r|^2$$

\end{metoda}
\vspace{0.5cm}

\begin{posledica}

Če je $f$ \textit{dvakrat zvezno odvedljiva} v okolici ničle $\alpha$, potem tangentna metoda za dovolj dober začetni približek $x_0$ vedno konvergira k $\alpha$.

\end{posledica}
\vspace{0.5cm}

\begin{izrek}

Naj bo funkcija $f$ na $I = [a, \infty)$ \textit{dvakrat zvezno odvedljiva, naraščajoča in ima ničlo na $\alpha \in I$}. Potem je $\alpha$ edina ničla na $I$ in za $\forall x_0 \in I$ tangentna metoda konvergira le k $\alpha$.

\end{izrek}
\vspace{0.5cm}

% *************************************************************************************************

\subsection{Metode brez $f'$}
\vspace{0.5cm}


\begin{metoda}[Sekantna metoda]

$$x_{r+1} = x_r - \frac{f(x_r)(x_r - x_{r-1})}{f(x_r) - f(x_{r-1})}$$
Konvergenca:
$$|e_{r+1}| \approx C \cdot |e_r| \cdot |e_{r-1}|$$

\end{metoda}
\vspace{0.5cm}

\begin{metoda}[Mullerjeva metoda]

Skozi točke $(x_r, f(x_r))$, $(x_{r-1}, f(x_{r-1}))$, \\$(x_{r-2}, f(x_{r-2}))$ potegnemo kvadratni polinom $y = p(x)$ in za $x_{r+1}$ vzamemo tisto ničlo polinoma $p$, ki je bližje $x_r$. \\

\noindent Konvergenca:
$$|e_{r+1}| \approx C \cdot |e_r| \cdot |e_{r-1}| \cdot |e_{r-2}|$$

\end{metoda}
\vspace{0.5cm}

\begin{metoda}[Inverzna interpolacija]

Zamenjamo vlogi $x$ in $y$ in vzamemo kvadratni polinom $x = \mathcal{L}(y)$, ki gre skozi točke $(x_r, f(x_r))$, $(x_{r-1}, f(x_{r-1}))$, $(x_{r-2}, f(x_{r-2}))$. Za $x_{r+1}$ vzamemo
$$x_{r+1} = \mathcal{L}(0).$$
Red konvergence je enak kot pri \textit{Mullerjevi metodi}.

\end{metoda}
\vspace{0.5cm}

% *************************************************************************************************

\pagebreak

% #################################################################################################

\section{SISTEMI LINEARNIH ENAČB}
\vspace{0.5cm}

% *************************************************************************************************

\subsection{Oznake in definicije}
\vspace{0.5cm}

\begin{definicija}

Sistem $n$ linearnih enačb z $n$ neznankami pišemo v obliki
$$Ax = b, ~~A \in \mathbb{R}^{n \times n} ~(\mathbb{C}^{n \times n}), ~x, b \in \mathbb{R}^n ~(\mathbb{C}^n).$$

\end{definicija}
\vspace{0.5cm}

\begin{definicija}

Skalarni produkt vektorjev $x$ in $y$ je enaka
\begin{enumerate}
	\item[a)] $x, y \in \mathbb{R}^n$:
	$$y^T x ~=~ \sum_{i=1}^n x_i y_i ~=~ \langle x, y \rangle ~=~ \langle y, x \rangle$$
	\item[b)] $x, y \in \mathbb{C}^n$:
	$$y^H x ~=~ \sum_{i=1}^n x_i \overline{y_i} ~=~ \langle x, y, \rangle ~=~ \overline{\langle y, x \rangle}$$
\end{enumerate}

\end{definicija}
\vspace{0.5cm}

\begin{definicija}

Množenje vektorja $x$ z matriko $A$:
\begin{enumerate}
	\item[a)] $$y_i ~=~ \sum_{k=1}^n a_{ik} x_k ~=~ \alpha_i^T x$$
	\item[b)] $$y ~=~ \sum_{i=1}^n x_i a_i$$
\end{enumerate}

\end{definicija}
\vspace{0.5cm}

\begin{definicija}

$A \in \mathbb{R}^{n \times n}$ je \textit{nesingularna}, če (ekvivalentno):
\begin{enumerate}
	\item[a)] $\det{(A)} \neq 0$
	\item[b)] obstaja inverz $A^{-1}$, da je $A^{-1}A = AA^{-1} = I$
	\item[c)] $rang{(A)} = n$
	\item[d)] za $\forall x \neq 0$ je $Ax \neq 0$
	\item[e)] $\ker{(A)} = \{x ~|~ Ax = 0\} = \{0\}$
\end{enumerate}

\end{definicija}
\vspace{0.5cm}

\begin{definicija}

Matrika je \textit{simetrično pozitivno definitna}, če $A = A^T$ in $x^T A x > 0$ za $x \neq 0$.

\end{definicija}
\vspace{0.5cm}

\begin{definicija}

Če za $x \neq 0$ velja $Ax = \lambda x$, je $\lambda$ lastna vrednost in $x$ lastni vektor. Vsaka matrika ima $n$ lastnih vrednosti, ki so ničle karakterističnega polinoma
$$p(\lambda) := \det{(A - \lambda I)}.$$

\end{definicija}
\vspace{0.5cm}

% *************************************************************************************************

\subsection{Vektorske in matrične norme}
\vspace{0.5cm}

\begin{definicija}

Vektorska norma je preslikava $\|.\|: \mathbb{C}^n \rightarrow \mathbb{R}$, da velja:
\begin{enumerate}
	\item[1)] Nenegativnost: 
	$$\|x\| \geq 0, ~~\|x\| = 0 \Leftrightarrow x = 0$$
	\item[2)] Homogenost:
	$$\|\alpha x\| = |\alpha| \cdot \|x\|$$
	\item[3)] Trikotniška neenakost:
	$$\|x+y\| \leq \|x\| + \|y\|$$
\end{enumerate}

\end{definicija}
\vspace{0.5cm}

\begin{definicija}

Matrična norma je preslikava $\|.\|: \mathbb{C}^{n \times n} \rightarrow \mathbb{R}$, da velja:
\begin{enumerate}
	\item[1)] $$\|A\| \geq 0, ~~\|A\| = 0 \Leftrightarrow A = 0$$
	\item[2)] $$\|\alpha A\| = |\alpha| \cdot \|A\|$$
	\item[3)] $$\|A + B\| \leq \|A\| + \|B\|$$
	\item[4)] Submultiplikativnost:
	$$\|A \cdot B\| \leq \|A\| \cdot \|B\|$$
	za $\forall A, B \in \mathbb{C}^{n \times n}$ in $\forall \alpha \in \mathbb{C}$.
\end{enumerate}

\end{definicija}
\vspace{0.5cm}

\begin{lema}

Če je za poljubno vektorsko normo $\|.\|_v$ definiramo 
$$\|A\| := \max_{x \neq 0} \frac{\|Ax\|_v}{\|x\|_v},$$
je to matrična norma.

\end{lema}
\vspace{0.5cm}

\begin{lema}

$$\|A\|_1 ~=~ \max_{j = 1, \ldots, n} \|a_j\|_1 ~=~ \max_{j = 1, \ldots, n} \sum_{i=1}^n |a_{ij}|$$

\end{lema}
\vspace{0.5cm}

\begin{lema}

$$\|A\|_2 ~=~ \sigma_1 (A) = \max_{i = 1, \ldots, n} \sqrt{\lambda_i (A^H A)}$$

\end{lema}
\vspace{0.5cm}

% *************************************************************************************************

% #################################################################################################

\end{document}