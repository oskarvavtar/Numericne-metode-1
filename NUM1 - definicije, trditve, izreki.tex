\documentclass[11pt]{article}
\usepackage[utf8]{inputenc}
\usepackage[slovene]{babel}

\usepackage{amsthm}
\usepackage{amsmath, amssymb, amsfonts}

\theoremstyle{definition}
\newtheorem{definicija}{Definicija}[section]

\newtheorem{lema}{Lema}
\newtheorem{trditev}{Trditev}
\newtheorem{izrek}{Izrek}
\newtheorem*{dokaz}{Dokaz}

\title{Numerične metode 1 - definicije, trditve in izreki}
\author{Oskar Vavtar}
\date{2020/21}

\begin{document}
\maketitle
\pagebreak
\tableofcontents
\pagebreak

% #################################################################################################

\section{NUMERIČNO RAČUNANJE}
\vspace{0.5cm}

% *************************************************************************************************

\subsection{Uvod}
\vspace{0.5cm}

\begin{definicija}[Napaka]

Pri numeričnem računanju izračunamo numerični približek za točno rešitev. Razlika med približkom in točno vrednostjo je \textit{napaka} približka. Ločimo \textit{absolutno} in \textit{relativno} napako.

\begin{itemize}
	\item absolutna napaka $=$ približek $-$ točna vrednost
	\item relativna napaka $=$ $\frac{\text{absolutna napaka}}{\text{točna vrednost}}$
\end{itemize}

Naj bo $x$ točna vrednost, $\hat{x}$ pa približek za $x$.

\begin{itemize}
	\item Če je $\hat{x} = x + d_a$, potem je $d_a = \hat{x} - x$ \textit{absolutna napaka}.
	\item Če je $\hat{x} = x(1 + d_r)$, potem je $d_r = \frac{\hat{x} - x}{x}$ \textit{relativna napaka}.
\end{itemize}

\end{definicija}
\vspace{0.5cm}

% *************************************************************************************************

\subsection{Premična pika}
\vspace{0.5cm}

\begin{definicija}

Velja fl$(x) = x(1 + \delta)$ za $|\delta| \leq u$, kjer je
$$u = \frac{1}{2}b^{1-t}$$
\textit{osnovna zaokrožitvena napaka}:

\begin{itemize}
	\item single: $u = 2^{-24} = 6 \times 10^{-8}$
	\item double: $u = 2^{-53} = 1 \times 10^{-16}$
\end{itemize}

\end{definicija}
\vspace{0.5cm}

\begin{izrek}

Če za število $x$ velja, da $|x|$ leži na intervalu med najmanjšim in največjim \textit{pozitivnim predstavljivim normaliziranim} številom, potem velja
$$\frac{|\text{fl}(x) - x|}{|x|} \leq u.$$

\end{izrek}
\vspace{0.5cm}

% *************************************************************************************************

\subsection{Občutljivost problema}
\vspace{0.5cm}

\begin{definicija}[Občutljivost]

Če se rezultat pri majhni spremembi argumentov (\textit{motnji} oz. \textit{perturbaciji}) ne spremeni veliko, je problem \textit{neobčutljiv}, sicer pa je \textit{občutljiv}.

\end{definicija}
\vspace{0.5cm}

\end{document}